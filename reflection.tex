\section{Introduction}\label{introduction}

\subsection{H2}\label{h2}

\subsubsection{H3}\label{h3}

\paragraph{H4}\label{h4}

\section{Disaster Data Recovery
Training}\label{disaster-data-recovery-training}

\subsection{H2}\label{h2-1}

\subsubsection{H3}\label{h3-1}

\paragraph{H4}\label{h4-1}

\paragraph{What? (description)}\label{what-description}

Each year the infrastructure team take part in data recovery training at
the IBM server site in Greenford, this is for multiple reasons; it
ensures that data being stored off site via Magnetic tape is intact and
all accounted for, to account for any variation in the recovery process
since the last recovery session and to ensure everyone knows how to
restore key systems as quickly as possible to reduce downtime a server
failure may cause.

\paragraph{What happened? Who was
involved?}\label{what-happened-who-was-involved}

During the two days I shadowed several members of the team to observe
each role and see how each persons dedicated knowledge is essential to
the recovery process. There were many steps involved with most being
dependent on several other aspects. The initial step involved was simply
a matter of loading the tapes to begin the data copy a process that took
a while but allowed time for other to begin spinning up servers and the
virtual machines that run upon them. As with most things in IT this did
not go smoothly with several unforeseen issues occurring, first and
foremost was with the contract with IBM where the external broadband
connection within the contract had lapsed resulting in no external
internet access which was required to connect to Crawley site.

There were also issues with configuring the virtual machines as they
failed to detect the hardware support for virtulaisation and required
some research and applying a work around found by the server architect.
There were also several issues with restoring some data to the server
once it had been started, however this was not critical data and was not
seen as immediate issue. At the end of the first day after starting at
around 9am and calling it a day at just past 6pm all critical systems
were back up and running with some extra systems started and most data
restored at which point it was decided it was time to head off and pick
up where we left off the following day. Under a real situation this
would not have been the situation however and the team would have worked
through the night to fully restore the system. In the evening we went
for a meal as a team and had time to talk and relax before the next day,
it was especially nice for me to be able to talk to members of the team
who I often would have not get a chance such as with someone who works
from home usually.

\paragraph{So what? (interpretation)}\label{so-what-interpretation}

Failures do happen, a key point I feel I could take away from this trip;
the key thing to remember is working on them logically and resolving
each issue one at time as any task can be broken down into smaller
issues and ultimately everything was completed with just one day
required to restore the critical components along with several others
for the entire European and emerging markets network.

IBM were able to fulfill mostly any need required and were quickly able
to provide the broadband access needed; this alone made it clear to me
performing these sessions were crucial to outline any issues in the
recovery process. Anything and everything can potentially go wrong and
it is critical to have processes and reviews of those processes in place
not just for large companies but any company as disasters can happen in
any situation.

Team work is important and this has been told to me since as early as I
can remember and with the couple of days I spent on the recovery session
it was plain to see this is the case. Everyone has there own
specialities and experiences, helping each other troubleshoot and
resolve issues. Using documentation of the process alone would
eventually get the systems back up and running but with the amount of
problems that can occur the time taken to resolve these issues would
cause a huge time delay and in a company that spans multiple time zones
every minute really does count.

\paragraph{What is most important/interesting/relevant/ useful aspect of
the
event/idea/situation?}\label{what-is-most-importantinterestingrelevant-useful-aspect-of-the-eventideasituation}

Everyone has their own skills and abilities and it wouldn't be feasible
to expect one person to be able to maintain and support all of them. The
issue arises if someone is unable to attend and therefore it is also
paramount that documentation of the process be produced by each member
of the team, doing so is also good way to act as a reminder when a
disaster does occur but documentation can quickly become outdated in the
field of IT. Performing these data recovery processes highlights any
missing information and anything that has changed and allows each member
to update anything as necessary and by keeping detailed and up to date
instructions can be a great help for everyone.

\paragraph{How can it be applied in the future? \#\#\#\# How is it
similar to/different from others? \#\#\#\# What next? (outcome) \#\#\#\#
What have I
learned?}\label{how-can-it-be-applied-in-the-future-how-is-it-similar-todifferent-from-others-what-next-outcome-what-have-i-learned}

I've learned that with lots of different aspects and areas in the field
of IT and computing it isn't feasible to be an expert in all of them,
however knowing the basics for each will of great help; with a basic
understanding of a system it's beneficial to know what is expected to be
entered in and receive by output

\subsubsection{}\label{section}

What? (description) What happened? Who was involved? So what?
(interpretation) What is most important/interesting/relevant/ useful
aspect of the event/idea/situation? How can it be explained? How is it
similar to/different from others? What next? (outcome) What have I
learned? How can it be applied in the future?

What to include

Here are some tips on what to include in your reflective writing:

Don't just describe -- explore and explain what happened. Be honest --
it's ok to admit to making mistakes as well as success. But you should
also show how you understand why things happen and what you are going to
do to improve. Be selective -- you don't have to write about everything
that happened, just key events or ideas. Look to the future -- reflect
on what happened in the past and how it will have an impact on future
ideas or activities.
